%% add stuff about Simon's? 
%% PPREP MENTORSHIP
%% Brown seminar 
% open mind editor
\documentclass[margin,line,pifont,palatino,courier]{res}

\usepackage{pifont}
\usepackage[latin1] { inputenc}
\usepackage{changepage}
%\topmargin .5in
%\oddsidemargin -.5in
%\evensidemargin -.5in
%\textwidth=6.0in
 \textheight=9.0in
%\itemsep=0in
%\parsep=0in
\usepackage{fancyhdr}
%\topmargin=0in
%\textheight=8.5in
\pagestyle{fancy}
\renewcommand{\headrulewidth}{0pt}
\fancyhf{}
%\cfoot{\thepage}
%\lfoot{\textit{\footnotesize Research Statement}}
%\rfoot{{\footnotesize Curriculum Vitae, Tomer D. Ullman, \thepage}}


\newenvironment{list1}{
  \begin{list}{\ding{113}}{%
      \setlength{\itemsep}{0in}
      \setlength{\parsep}{0in} \setlength{\parskip}{0in}
      \setlength{\topsep}{0in} \setlength{\partopsep}{0in}
      \setlength{\leftmargin}{0.17in}}}{\end{list}}
\newenvironment{list2}{
  \begin{list}{$\bullet$}{%
      \setlength{\itemsep}{0in}
      \setlength{\parsep}{0in} \setlength{\parskip}{0in}
      \setlength{\topsep}{0in} \setlength{\partopsep}{0in}
      \setlength{\leftmargin}{0.2in}}}{\end{list}}
\newenvironment{list3}{
  \begin{list}{$\bullet$}{%
      \setlength{\itemsep}{0in}
      \setlength{\parsep}{0in} \setlength{\parskip}{0in}
      \setlength{\topsep}{0in} \setlength{\partopsep}{0in}
      \setlength{\leftmargin}{0.2in}}}{\end{list}}
\begin{document}

\name{Tomer D. Ullman \vspace*{.1in}}

\begin{resume}

\section{\sc Contact Information}

\vspace{.05in}
\begin{tabular}{@{}p{2.75in}p{2in}}
Harvard University  \\
Department of Psychology                        & \verb+tullman@fas.harvard.edu+\\
William James Hall Room 1320            & \verb+http://www.tomerullman.org+\\
33 Kirkland St., Cambridge, MA 02138 USA               & \\
\end{tabular}

\section{\sc Position}
Assistant Professor, Harvard Department of Psychology. Member of the Center for Brains, Minds, and Machines. Co-PI for the Science of Intelligence group at Harvard. 

\section{\sc Research Interests}
Computational cognitive modeling, intuitive theories, child development, probabilistic programming, folk physics, folk psychology, common sense, machine learning, theory-of-self

\section{\sc Education}


{\bf Massachusetts Institute of Technology and Harvard University}\\
\vspace*{-.1in}
\begin{list1}
\item[] Postdoctoral associate at the Center for Brains, Minds \& Machines (2016-2019)
\end{list1}

{\bf Massachusetts Institute of Technology}\\
\vspace*{-.1in}
\begin{list1}
\item[] Ph.D. in Brain and Cognitive Sciences (2008-2015)

\begin{list2}
\vspace*{.05in}
\item Dissertation Topic:  On the Nature and Origin of Intuitive Theories
\end{list2}
\end{list1}

{\bf Hebrew University of Jerusalem}\\
\vspace*{-.1in}
\begin{list1}
\item[] B.S.~in Physics and Cognitive Science (double major, 2004-2008)

\begin{list2}
\vspace*{.05in}
\item \textit{Magna Cum Laude}
\end{list2}

\end{list1}

\begin{adjustwidth*}{-3.3cm}{-2cm}
\textbf{Grants}\\
\noindent\rule{8cm}{0.4pt}

\end{adjustwidth*}

\section{\sc } 

DARPA MCS (Machine Common Sense), June 2019 -- June 2023, \$12,100,000, \\
``Building machine common sense the human way'', Co-PI with six other investigators from IBM,
MIT, Stanford, and Harvard. Harvard component, supporting Ullman, Spelke, and
several RAs and postdoctoral researchers, is \$1,080,000.\\

John Templeton Foundation, Febuary 2020 -- April 2022, \$240,000, \\
``Play, a computational perspective''.\\

Templeton Experience Project, June 2015 -- June 2017, \$90,000, \\
``Computational models of the intuitive theory of transformative experiences'', with Josh Tenenbaum (PI).

\begin{adjustwidth*}{-3.3cm}{-2cm}
\textbf{Publications}\\
\noindent\rule{8cm}{0.4pt}

\end{adjustwidth*}

\section{\sc Manuscripts under Revision}

Kryven, M., Ullman, T.D., Cowan, W., and Tenenbaum, J.B., Strategy or luck: Intuitive theories of attributed intelligence.

Sosa, F. A., T. D. Ullman, J. B. Tenenbaum, S. J. Gershman, and T. Gerstenberg. Moral dynamics: Grounding moral judgment in intuitive physics and intuitive psychology.

Paul LA, Ullman, T.D., Tenenbaum, J.B., Reverse engineering a self

Ullman, T.D., Kosoy, E., Tenenbaum J.B., and Spelke, E., Preschoolers' understanding of heavy and light: Inference and prediction. 

\section{\sc Submitted Manuscripts}

S Liu, T Ullman, J Tenenbaum, E Spelke, Dangerous ground: Thirteen-month-old infants are sensitive to peril in other people's actions

\section{\sc Manuscripts in Prep}
%% move up 
Ullman, T.D., Xu, Y., and Goodman, N.D., The pragmatics of spatial language. 

\section{\sc Journal articles}

Ullman, T.D. and Tenenenbaum, J.B. (to appear), Bayesian models of conceptual development: Learning as building models of the world. \textit{Annual Review of Developmental Psychology}.


Ullman, T.D. (2020) Heroes of our own story: Self-image and rationalizing in thought experiments. \textit{Behavioral and Brain Sciences} 43.

McCoy, J. and Ullman, T. (2019) Transformative Decisions and Their Discontents. Part of a symposium on L.A. Paul's "Transformative Experience". \textit{Rivista Internazionale di Filosofia e Psicologia}, 10(3), 339 - 345.

Bonawitz E., Ullman, T.D., Gopnik, A., and Tenenbaum, J.B. (2019), Sticking to the evidence? A Computational and behavioral case Study of micro-theory change in the domain of magnetism, \textit{Cognitive Science}, 43(8), e12765.

Ullman, T.D. and McCoy, J.P. (2019), Judgments of effort for magical violations of intuitive physics. (2019) \textit{PloS one}, 14(5), e0217513.

McCoy, J.P., and Ullman, T.D., A Minimal Turing Test. (2018). A Minimal Turing Test. \textit{Journal of Experimental Social Psychology}, 79, 1-8.

Gerstenberg, T., Ullman, T.D., Nagel, J., Kleiman-Weiner, M., Lagnado, D., and Tenenbaum, J.B. (2018), Lucky or clever? From changed expectations to attributions of responsibility. \textit{Cognition.}

Liu, S., Ullman, T.D., Tenenbaum J.B., and Spelke, E. (In press) 10-month-olds infer the value of goals from the costs of actions. \textit{Science}, \textit{358}(6366), 1038-1041.

Ullman, T. D., Spelke, E. S., Battaglia, P., and Tenenbaum, J. B. (2017), Mind Games: Game Engines as an Architecture for Intuitive Physics. \textit{Trends in Cognitive Science, 21}(9), 649--665.

Ullman, T. D., Stuhlm{\"u}ller, A., Goodman, N.D., and Tenenbaum, J. B. (2017), Learning physical parameters from dynamic scenes. \textit{Cognitive Psychology.}

Lake, B. M., Ullman, T. D., Tenenbaum, J. B., and Gershman, S. J. (2017), Building machines that learn and think like people. \textit{Behavioral and Brain Sciences,} 1--101.

Hamlin, J. K., Ullman, T. D., Tenenbaum, J. B., Goodman, N. D., and Baker, C. L. (2013), The mentalistic basis of core social cognition: Experiments in preverbal infants and a computational model. \textit{Developmental Science 16}(2), 209-226.

Ullman, T. D., Goodman, N. D., and Tenenbaum, J. B. (2012), Theory learning as stochastic search in the language of thought. \textit{Cognitive Development 27}(4), 455--480.

Goodman, N. D., Ullman, T. D., and Tenenbaum, J. B. (2012), Learning a theory of causality. \textit{Psychological Review, 118}(1), 110. 

\section{\sc Book Chapters}

Ullman, T.D. and Zimmerman, S., Models of transformative decision making, (2020) in Becoming Someone New: Essays on Transformative Experience, Choice, and Change, eds. Enoch Lambert and John Schwenkler, Oxford University Press.


Ullman, T. D., McCoy, J. P., and Paul, L. A., (2019), Modal Prospection. Metaphysics and Cognitive Science, eds. Alvin Goldman and Brian McLaughlin. Oxford University Press (US).


\section{\sc Peer Reviewed Conference Proceedings}

Smith, K. A., Mei, L., Yao, S., Wu, J., Spelke, E., Tenenbaum, J. B., and Ullman, T. D. (2020). The fine structure of surprise in intuitive physics: when, why, and how much?. In \textit{Proceedings of the 42nd Annual Meeting of the Cognitive Science Society.}

Gjata, N., Ullman, T. D., Spelke, E. S., and Liu, S. (2020). Look before you leap: Quantitative tradeoffs between peril and reward in action understanding. In \textit{Proceedings of the 42nd Annual Meeting of the Cognitive Science Society.}

Shu, T., Kryven, M., Ullman, T.D., and Tenenbaum, J.B. (2020). Adventures in Flatland: Perceiving Social Interactions Under Physical Dynamics In \textit{Proceedings of the 42nd Annual Meeting of the Cognitive Science Society.}

Smith, K.*, Mei, L.*, Yao, S., Wu, J., Spelke, E., Tenenbaum, J.B., and Ullman, T.D., (2019) Modeling expectation violation in intuitive physics with coarse probabilistic object representations. \textit{Advances in Neural Information Processing Systems}.

Ullman, T. D., Kosoy, E., Yildrim, I., Soltani, A., Siegel, X., Tenenbaum J.B., and Spelke, E.,(2019), \textit{Draping an Elephant: Uncovering Children's Reasoning About Cloth- Covered Objects Proceedings of the 41st Annual Conference of the Cognitive Science Society.}

Ullman, T. D., Alonso-Diaz, S., Ferringo, S., Zahid, S., and Kidd, C. (2017), Weight matters: The role of physical weight in non-physical language across age and culture. \textit{Proceedings of the $39^{th}$ Annual Conference of the Cognitive Science Society.}

Liu, S., Ullman, T.d., and McCoy, J.P., (2019), People's perception of others' risk preferences. \textit{Proceedings of the 41st Annual Conference of the Cognitive Science Society.}

Liu, S., Ullman, T. D., Tenenbaum, J. B., and Spelke, E. S. (2017), What's worth the effort: Ten-month-old infants infer the value of goals from the costs of actions. \textit{Proceedings of the $39^{th}$ Annual Conference of the Cognitive Science Society.}

Kryven, M., Ullman, T. D., Cowan, W., and Tenenbaum, J. B. (2017), Thinking and guessing: Bayesian and empirical models of how humans search. \textit{Proceedings of the $39^{th}$ Annual Conference of the Cognitive Science Society.}

Chang, M. B., Ullman, T. D., Torralba, A., and Tenenbaum, J. B. (2017), A compositional object-based approach to learning physical dynamics. \textit{International Conference on Learning Representations (ICLR).}

Ullman, T.D., Xu, Y. \& Goodman, N.D. (2016), The Pragmatics of spatial language. \textit{Proceedings of the $38^{th}$ Annual Conference of the Cognitive Science Society.}

Ullman, T.D., Siegel, M., Tenenbaum, J.B. \& Gershman, S.J. (2016), Coalescing the vapors of human experience into a viable and meaningful comprehension. \textit{Proceedings of the $38^{th}$ Annual Conference of the Cognitive Science Society.}

Kryven, M., Ullman, T.D., Cowan, W. \& Tenenbaum, J.B. (2016), Outcome or strategy? A Bayesian model of intelligence attribution. \textit{Proceedings of the $38^{th}$ Annual Conference of the Cognitive Science Society.}

Gerstenberg, T., Ullman, T. D., Kleiman-Weiner, M., Lagnado, D. A. \& Tenenbaum, J. B. (2014), Wins above Replacement: Responsibility attributions as counterfactual replacements. \textit{Proceedings of the $36^{th}$ Annual Conference of the Cognitive Science society.}

Ullman, T.D., Stuhlm{\"u}ller A., Goodman, N.D. \& Tenenbaum, J.B. (2014), Learning physics from dynamical scenes. \textit{Proceedings of the $36^{th}$ Annual Conference of the Cognitive Science society.}

Bonawitz E., Ullman, T.D., Gopnik, A. \& Tenenbaum, J.B. (2012), Sticking to the evidence? A Computational and behavioral case Study of micro-theory change in the domain of magnetism, \textit{International Conference Developmental Learning and Epigenetic Robotics; best paper award: experiment combined with computational model.}

Ullman, T.D.*, McCoy, J.P.*, Stuhlmuller, A., Gerstenberg, T. \& Tenenbaum J.B. (2012), Why blame Bob? Probabilistic generative models, counterfactual reasoning, and blame attribution. \textit{Proceedings of the $33rd$ Annual Conference of the Cognitive Science Society.}

Ullman, T.D., Goodman, N.D. \& J. B. Tenenbaum (2010), Theory acquisition as stochastic search. \textit{Proceedings of the $32nd$ Annual Conference of the Cognitive Science Society.}

Ullman, T.D., Baker, C.L., Macindoe, O., Evans, O., Goodman, N.D. \& Tenenbaum, J.B. (2010), Help or hinder: Bayesian models of social goal inference. \textit{Advances in Neural Information Processing Systems (Vol. 22, pp. 1874-1882).}

Goodman, N.D., Ullman, T.D. \& Tenenbaum, J.B. (2009), Learning a theory of causality \textit{Proceedings of the $31st$ Annual Conference of the Cognitive Science Society.}

\vspace{20pt}
\begin{adjustwidth*}{-3.3cm}{-2cm}

\hspace{-3.8em}\textbf{Invited Talks and Presentations}\\
\hspace*{-3.8em}\noindent\rule{8cm}{0.4pt}

\end{adjustwidth*}

\section{\sc }

Models of Core Knowledge\\
Center for Human-Compatible AI at UC Berkeley (CHAI), 2020

Discussant in `Is that so? How children evaluate claims and conjectures'\\
Child Development Society (CDS) Meeting, Louisville, KT, 2019.

Computational Models of Core Intuitive Physics,\\
Yale Current Works series, Yale, CN, 2019.

Neuro-symbolic Computing and Machine Common Sense\\
AI Research Week, IBM, MA, 2019.

Computational Models of Core Intuitive Physics,\\
Facebook Workshop on Understanding Human and Machine Intelligence, NYC, NY,\\
2019.

Reverse Engineering a Self,\\
Formal and Experimental Conference, Northeastern, MA, 2019

Thinking New Thoughts,\\
Workshop on Possibility and Value, Radcliffe Institute for Advanced Studies, MA,\\
2019.

Evaluating Future Selves,\\
Workshop on Imagination, Simulation, and the Self, Tufts, MA, 2018.

Physics Meets Development,\\
CVPR workshop on Vision Meets Cognition, Salt Lake City, UT, 2018.

Canonical Mass: Preschoolers Expectations of Dynamic Variables for Solid Objects,\\
Society for Research in Child Development, Austin, TX. 2017.

Modal Imagination,\\ Ranch Metaphysics Workshop, Tucscon, AZ. 2017.

People and Things,\\ Current Work in Developmental Psychology Colloquium. Boston College, MA. 2016. 

Development, Psychology, Physics. \\DeepMind Technologies, London. 2016. 

Modal Prospection \\ Philosophy Seminar. Rutgers University, NJ. 2016.

Imagining and Evaluating Possible Future Selves, \\ $42nd$ Meeting of the Society for Philosophy and Psychology. Austin, TX. 2016.

Computational Cognitive Science,\\ Interdisciplinary College on AI, Germany. 2016. 

Probabilistic Programming,\\ Interdisciplinary College on AI, Germany. 2016. 

Effort as a Bridge Across Action and Action Understanding,\\
The $20^{th}$ International Congress on Infant Studies. New Orleans, LA. 2016.

Children's Learning as Stochastic Search, \\Society for Research in Child Development. Philadelphia, PA. 2015.

Theories, Imagination, and the Generation of New Ideas, \\Center for Brains, Minds and Machines Summer School. Woods Hole, MA. 2015.

Probabilistic Programming Tutorial, \\ Center for Brains, Minds and Machines Summer School. Woods Hole, MA. 2015.

Theories of Physics, \\More on Development. Columbus, OH. 2015.

Modeling a Theory of the Self, \\Workshop on Transformative Experiences. Chicago, IL. 2015.

Deep Thoughts: The Value of Understanding,\\ Commentator at $41^{th}$ Meeting of the Society for Philosophy and Psychology. Duke University, NC. 2015.

Wins Above Replacement: Responsibility Attributions as Counterfactual Replacement,\\ $40^{th}$ Meeting of the Society for Philosophy and Psychology. Vancouver, Canada. 2014.

Theories, Imagination, and the Generation of New Ideas, \\Center for Brains, Minds and Machines Summer School. Woods Hole, MA 2014.

Probabilistic Programming Tutorial, \\ Center for Brains, Minds and Machines Summer School. Woods Hole, MA. 2014.

Learning Physics from Dynamic Scenes, \\$36^{th}$ Annual Meeting of the Cognitive Science Society. Quebec, Canada. 2014.

Theories, Imagination, and the Generation of New ideas,\\ Pre-conference debate at Child Development Society Meeting. Memphis, TN. 2013.

Help or Hinder? Bayesian Models of Social Goal Inference,\\ Simons Center, MIT. 2010.

Why Blame Bob? Probabilistic Generative Models and Blame Attribution,\\ $34^{th}$ Annual Meeting of the Cognitive Science Society. Sapporo, Japan. 2012.

Theory Learning as Stochastic Search,\\ $32nd$ Annual Meeting of the Cognitive Science Society. Portland, OR. 2010.

Help or Hinder? Bayesian Models of Social Goal Inference,\\ Machine Learning Summer School, Cambridge, UK. 2009.

Help or Hinder? Bayesian Models of Social Goal Inference,\\ $23rd$ Annual Conference on Neural Information Processing Systems. Vancouver. 2009.

\begin{adjustwidth*}{-17.8cm}{-2cm}

\hspace{-3.8em}\textbf{Other}\\
\hspace*{-3.8em}\noindent\rule{8cm}{0.4pt}

\end{adjustwidth*}

\section{\sc Teaching Experience and Outreach}

\begin{tabular}{@{}p{0.4in}p{0.3in}p{4in}}

2020 & & Instructor, `Imagination, Pretense, and Make-Believe Worlds' (Harvard PSY1340) \\
2020 & & Instructor, `Decisions Big and Small' (Harvard PSY1322) \\
2019 & & Guest lecturer, Harvard Developmental Seminars\\
2018 & & Guest lecturer, Harvard Developmental Seminars\\
2017 & & Lecturer and Teaching Assistant, CBMM summer school\\
2015 & & Lecturer and Teaching Assistant, CBMM summer school\\
2014 & & Lecturer and Teaching Assistant, CBMM summer school\\
2012 & & Teaching Assistant, Topics in early childhood cognition (MIT 9.85)\\
2011 & & Teaching Assistant, Cognitive processes (MIT 9.65)\\
2010 & & Planning committee member, Cambridge Science Festival\\
2009 & & Presenter and volunteer at Neuroscience Day, Museum of Science
\end{tabular}


\section{\sc Honors and Awards}

\begin{tabular}{@{}p{0.8in}p{4in}}
2020 & Harvard University Special Commendation: Extraordinary Teaching in Extraordinary Times \\  
2011 & ICDL Best Paper Award: Experiment with computational model  \\
2011 & MIT Continued Dedication to Teaching award  \\
2010 & MIT Excellence in Teaching award  \\
2010 & National Science Foundation (NSF) fellowship\\
2009 & Singleton Graduate Fellowship \\
2009 & National Science Foundation (NSF) honorable mention\\
2004-2007 & Hebrew University of Jerusalem Scholarships of Excellence
\end{tabular}


\section{\sc Service}

Reviewer (partial): Cognition, Cognitive Psychology, Cognitive Development, Cognitive Science, Cognitive Research: Principles and Implications, Developmental Psychology, Developmental Science, JEP: General, Nature Human Behavior, Psychological Review, Proceedings of the Royal Society B, Topics in Cognitive Science, Philosophical Psychology, Neural Information Processing Systems (NeurIPS), The Annual Conference of the Cognitive Science Society (CogSci), AAAI, PLoS ONE

Ad-hoc Editor for Open Mind 

Grant Reviews: Templeton Foundation, Israel Science Foundation (ISF)

Ongoing organizer of Machine Common Sense reasoning meetings (intuitive psychology and general track, 2019-2020)

Co-organizer of ``The Origins of Commonsense in Humans and Machines'' Workshop, Cogsci (2020)

Co-organizer of Lorentz Center ``Developing Models of the Word'' Workshop, Leiden,\\
Netherlands (2019)

Co-organizer of Cognition, Brain, Behavior lunch seminar series, Harvard (2019, 2020)

PC Member AAAI main track

Organizer of ``More on Development (MOD)'' (Ohio, 2015)

Co-Organizer of Child Development Society pre-conference on ``computational cognitive models and cognitive development'' (2014)

\section{\sc Students Mentored (partial)}

Konstantina Katsimeni (2020)

Julian De Freitas (2020)

Enosa Ogbeide (2019)

Nensi Gjata (2019)

Shari Liu (2018-2020)

Felix Sosa (2017--)

Cameron Nieters (2017)

Michael Chang (2016, 2017)

Eliza Kosoy (2016, 2017)

Heather Tarr (2016)

Alexandra Wheeler (2016)

Samuel Zimmerman (2016)

Marta Kryven (2016)


\section{\sc Membership}

Association for Psychological Science (APS)

Cognitive Science Society (CSS)

Society for Research in Child Development (SRCD)

Cognitive Development Society (CDS)

Society for Philosophy and Psychology (SPP)

Society for Personality and Social Psychology (SPSP)


\section{\sc References}

{\bf Josh Tenenbaum}, Professor,
Brain and Cognitive Sciences, MIT,
 \texttt{jbt@mit.edu}

{\bf Laura Schulz}, Professor,
Brain and Cognitive Sciences, MIT,
 \texttt{lschulz@mit.edu}

{\bf Elizabeth Spelke}, Professor,
Psychology, Harvard,
\texttt{spelke@wjh.harvard.edu}

{\bf Noah Goodman}, Professor,
Psychology, Stanford University,
 \texttt{ngoodman@stanford.edu}

\end{resume}
\end{document}
